\documentclass[a4paper]{article}
\usepackage[utf8]{inputenc}
\usepackage[T1]{fontenc}
\usepackage{graphics}
\usepackage{libertine}
\usepackage[french]{babel}
%\usepackage{tango}
%\usepackage[colorlinks=true]{hyperref}

%\hypersetup{urlcolor=DarkChocolate}

\title{Rapport d’activité}
%\author{\scalebox{0.35}{ \includegraphics{logo_azae}}}
\author{Thomas Clavier}


\begin{document}

\maketitle

Durant ces 3 dernières années, mon activité professionnel dans son ensemble à été une symbiose entre monde industriel et universitaire au service de l'IUT. 
Nombre de mes activités se sont retrouvés expérimentés en formation professionnel avant d'arriver avec les élèves universitaire.
Voici brièvement le détail de ce qui a été fait.

\section{Activité d'enseignement}
\subsection{Extra universitaire}
Azaé, ma société est organisme de formation et la transmission du savoir représente la majorité de notre activité.

Nous faisons de l'innovation organisationnelle, nous aidons nos clients à devenir plus agile. Mais livrer plus vite, plus souvent tout en écoutant mieux ses propres clients, cela passe par une parfaite maitrise de l'artisanat du logiciel.

Ce que nous entendons par <<coaching et mentoring en artisanat logiciel>>, c'est aider nos client à s'approprier les pratiques suivantes : 
\begin{itemize}
  \item code propre, plus que des règles d'esthétiques c'est une rigueur et une philosophie de travail capable de considérablement réduire les couts de maintenance.
  \item parfaite maitrise des outils de gestion de version
  \item mise en place d'une stratégie de tests 
  \item systématiser la pratique du test driven développement
  \item cycle de production du logiciel
  \item et maitrise des outils facilitant le déploiement en continu.
\end{itemize}

Pour aider nos clients à maitriser toutes ces facettes du développement logiciel nous mettons en œuvre de nombreuses techniques pédagogiques.

Remettre en cause les managers et les pousser à adopter de nouvelles postures demande une réel innovation pédagogique afin de les motiver dans leurs apprentissages tout en préservant leur autonomie.

\subsection{Universitaire}
\subsubsection{Année 1}

TP de base de données
Initiation à git l'outil de gestion de version de code source : cours TP, TD
Initiation à l'artisanat logiciel : code propre, test unitaire, et TDD.
classe inversé
Initiation à l'agilité : cours, TP, TD.
Mise en place du projet agile : objectif pédagogique suivant : 
- mise en application des enseignements du semestre dans un format startup
- initiation par la pratique à la gestion de projet agile, 
- cohésion de la promotion dans son ensemble,
- confronter les étudiants à de vrais porteur de projet.

Cette année le projet agile à regroupé les étudiants d'info et les étudiants de GEA option création d'entreprise : mixage des étudiants entre filière.

Soutenance sous forme de salon avec communiqué de presse.

Pour se faire j'ai largement utilisé mon réseau professionnel pour impliquer de vraies porteurs de projets. 
J'ai initier un partenariat entre AUDACE, la BGE et l'IUT.

Le suivi officieux des étudiants en stage est simplifié par mon réseau professionnel.

Formation des enseignants aux techniques de clean code, TDD, gestion de version, test unitaire.

Participation avec une double casquette aux journées de l'entrepreunariat au seins de l'IUT.

\section{Activité de recherche / valorisation}


Implication dans la réflexion d'un DU <<Lean startup>>
Implication dans la création d'une licence pro <<Développement mobile>> 
Coding gouter : initiation de l'informatique aux enfants entre 5 et 18 ans.
Meetup lean startup auquel je retrouve quelques anciens étudiants maintenant porteur de projet et des enseignants chercheur en phase de création d'entreprise.
Meetup Docker
Startup weekend
Conférencier dans divers conférences agiles (agile tours lille, agile laval, mixit, lean kanban france, agile open france, etc.)

\section{Implication dans la vie de l'IUT}
M3301, M3302



\end{document}

