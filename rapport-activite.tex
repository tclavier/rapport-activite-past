\documentclass[a4paper]{article}
\usepackage[utf8]{inputenc}
\usepackage[T1]{fontenc}
\usepackage{graphics}
\usepackage{libertine}
\usepackage[french]{babel}
%\usepackage{tango}
%\usepackage[colorlinks=true]{hyperref}

%\hypersetup{urlcolor=DarkChocolate}

\title{Rapport d’activité}
%\author{\scalebox{0.35}{ \includegraphics{logo_azae}}}
\author{Thomas Clavier}


\begin{document}

\maketitle

Durant ces 3 dernières années, mon activité professionnel dans son ensemble à été une symbiose entre monde industriel et universitaire au service de l'IUT. 
Nombre de mes activités se sont retrouvés expérimentés en formation professionnel avant d'arriver avec les élèves universitaire.
Voici brièvement le détail de ce qui a été fait.

\section{Activité d'enseignement}
\subsection{Extra universitaire}
Azaé, ma société est organisme de formation et la transmission du savoir représente la majorité de notre activité.

Nous faisons de l'innovation organisationnelle, nous aidons nos clients à devenir plus agile. Mais livrer plus vite, plus souvent tout en écoutant mieux ses propres clients, cela passe par une parfaite maitrise de l'artisanat du logiciel.

Ce que nous entendons par <<coaching et mentoring en artisanat logiciel>>, c'est aider nos client à s'approprier les pratiques suivantes : 
\begin{itemize}
  \item code propre, plus que des règles d'esthétiques c'est une rigueur et une philosophie de travail capable de considérablement réduire les couts de maintenance.
  \item parfaite maitrise des outils de gestion de version
  \item mise en place de stratégies de tests 
  \item systématiser la pratique du test driven développement (TDD)
  \item maîtrise du cycle de production du logiciel
  \item et maitrise des outils facilitant le déploiement en continu.
\end{itemize}

Pour aider nos clients à maitriser toutes ces facettes du développement logiciel nous mettons en œuvre de nombreuses techniques pédagogiques.

De la même façon, remettre en cause les managers et les pousser à adopter de nouvelles postures demande une réelle innovation pédagogique afin de les motiver dans leurs apprentissages tout en préservant leur autonomie. 

\subsection{Universitaire}

\subsubsection{Première année}
\begin{itemize}
  \item TP de base de données
  \item TP tests unitaires
  \item Premier projet agile sur un groupe de S3
\end{itemize}

\subsubsection{Seconde année}
\begin{itemize}
  \item Initiation à l'outil de gestion de version de code source : <<git>>. Cours TP, TD pour toute la promotion de S2 (4 groupes + 1 groupe décalé)
  \item Initiation à l'artisanat logiciel (code propre, test unitaire, et TDD) : cours, TP, TD pour toute la promotion de S2 (4 groupes + 1 groupe décalé).
  \item Initiation à l'agilité : cours, TD.
  \item Administration système et réseau: cours, TP, TD en semestre décalé (1 groupe)
  \item projet agile avec tous les S3 (4 groupes + 1 groupe décalé)
\end{itemize}

\subsubsection{Troisième année}
\begin{itemize}
  \item Initiation à l'outil de gestion de version de code source git, généralisation aux formations continues
  \item Initiation à l'artisanat logiciel (code propre, test unitaire, et TDD) : généralisation aux formations continues
  \item Initiation à l'agilité : cours, TD.
  \item Généralisation du projet agile à toutes les promotions (Formation initial, formation continue, décalés, etc.) sous 2 formes : en début de S3 puis en fin de S4.
\end{itemize}

\subsubsection{Projet agile en détail}

Le projet agile à été construit de façon incrémentale, d'abord 1 groupe en S3, puis toute la promotion de S3 et enfin l'ensemble des promotions en S3 et S4.

Ces 2 projets par an que font maintenant les étudiants couvrent les objectifs pédagogique suivant : 
\begin{itemize}
\item Replonger rapidement les étudiants dans la programmation après 2 mois de vacances.
\item l'initiation par la pratique à la gestion de projet agile, 
\item la cohésion de la promotion dans son ensemble,
\item la mise en application des enseignements du semestre dans un format startup
\item confronter les étudiants à de vrais clients porteurs de projets.
\item mélanger les étudiants d'informatique et de GEA dans un projet commun afin de découvrir concrètement le métier des autres.
\item savoir vendre et défendre un projet devant du public 
\end{itemize}

La soutenance se faisant sous forme de salon avec communiqué de presse, cela apporte une certaine diversité dans les visiteurs, tout en participant au rayonnement de l'IUT au près du public.

Pour mener à bien cette activité j'ai largement utilisé mon réseau professionnel afin d'impliquer de vrais porteurs de projets. 
J'ai d'ailleurs initier un partenariat entre AUDACE, la BGE haut de France et l'IUT.

Enfin, le projet agile de fin de S4 est un véritable outil de communication, faire venir des porteurs de projets dans l'IUT durant une semaine et faire venir les journalistes pour le salon de soutenance, c'est un bon moyen de recruter de futur employeurs et de faire connaitre le vrai niveau de nos étudiants.

\subsubsection{}

\section{Activité de recherche / valorisation}

\begin{itemize}
  \item Implication dans la réflexion d'un DU <<Lean startup>>
  \item Implication dans la création d'une licence pro <<Développement mobile>> 
  \item Organisation des Coding gouter : initiation de l'informatique aux enfants entre 5 et 18 ans.
  \item Organisation des Meetup lean startup durant lesquels je retrouve quelques anciens étudiants maintenant porteur de projet et des enseignants chercheur en phase de création d'entreprise.
  \item Organisation des Meetup Docker
  \item Participation à quelques Startup weekend
  \item Conférencier dans divers conférences (agile tour lille, agile laval, mixit, lean kanban france, agile open france, agile france, etc.)
\end{itemize}

Mes clients étant nombreux et majoritairement sur a région, il n'est pas rare de croiser des étudiants en stage durant l'une ou l'autre de mes visites, c'est l'occasion d'échanger sur le stage, le stagiaire et l'IUT.

Mon implication dans la vie de l'IUT ne s'arrête pas là, je participe avec une double casquette aux journées de l'entrepreneuriat organisé par l'IUT.

Enfin je suis responsable des modules pédagogique M3301, M3302 et M4XXX.

\section{Bilan}

Que ce soit au niveau enseignement, via la formation des enseignants aux différentes techniques d'artisanat logiciel, ou par ma charge pédagogique, mon investissement dans l'IUT est réel. 
Enfin je continu à développer mon réseau professionnel et c'est avec plaisir que je le met au service de l'IUT.

\end{document}

