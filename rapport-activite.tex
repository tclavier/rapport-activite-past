\documentclass[a4paper]{article}
\usepackage[utf8]{inputenc}
\usepackage[T1]{fontenc}
\usepackage{graphics}
\usepackage{libertine}
\usepackage[french]{babel}
\usepackage{tabularx}
%\usepackage{tango}
%\usepackage[colorlinks=true]{hyperref}

%\hypersetup{urlcolor=DarkChocolate}

\title{
  \begin{tabularx}{\linewidth}{l}
    \hline\hline
    \\
    \Huge Rapport d’activité 2013/2019 \\
    \\
    \Large Thomas Clavier, \normalsize Université lille / Azaé \\
    \\
    \hline\hline
  \end{tabularx}
}
%\author{\scalebox{0.35}{ \includegraphics{logo_azae}}}
\author{}
\date{}


\begin{document}

\maketitle

Durant ces 6 dernières années, mon activité professionnelle dans son ensemble a été une symbiose entre monde industriel et universitaire au service de l'IUT.
Voici brièvement le détail de ce qui a été fait.

\section{Activité}
\subsection{Extra universitaire}
Chez Azaé, nous faisons de l'accompagnement et du conseil auprès des directions des systèmes d'information des entreprises. Nos clients sont nombreux et pas uniquement régionaux. Ils me permettent de rapporter dans l'IUT les pratiques et savoir faire les plus usités dans le monde de l'entreprise afin de préparer nos étudiants aux techniques modernes qui les attendent à la sortie de l'IUT.

Nous faisons de l'innovation organisationnelle, de la transformation d'organisation, nous aidons nos clients à devenir plus agiles. Livrer plus vite, plus souvent tout en écoutant mieux leurs propres clients, cela passe par une parfaite maîtrise de l'artisanat du logiciel.

Ce que nous entendons par <<artisanat du logiciel>>, c'est la culture de l'excellence du développement logiciel, à savoir :
\begin{itemize}
  \item code propre, plus que des règles d'esthétique c'est une rigueur et une philosophie de travail capables de considérablement réduire les coûts de maintenance.
  \item parfaite maîtrise des outils de gestion de version,
  \item mise en place de stratégies de tests,
  \item systématisation de la pratique du test driven développement (TDD),
  \item maîtrise du cycle de production de valeur du logiciel (jusqu'à l'exploitation),
  \item maîtrise de l'ensemble de ses outils.
\end{itemize}

Côté management, nous observons une remise en cause de la vision actuelle du management pour adopter de nouvelles postures et un nouvel état d'esprit entre autre avec l'arrivée en force de l'agilité. Ce management bien plus adapté aux génération Y et Z demande une bien plus grande autonomie des développeurs. Il me semble donc très important de préparer nos étudiants à des environnements plus agiles.

Dernier point sur lequel nous accompagnons nos clients, la gestion produit. Gérer un projet sans gâchis pour apprendre à sortir un produit adapté aux clients de façon efficiente avec le lean startup, ou le partage et le découpage avec le story mapping et l'event storming.

En bref, notre expertise nous permet d'accompagner nos clients sur les 3 métiers suivant :
\begin{itemize}
  \item Artisanat logiciel
  \item Management agile
  \item Product Owner, gestion produit.
\end{itemize}

\subsection{Universitaire}

\subsubsection{Première année}
\begin{itemize}
  \item TP de base de données (1 groupe)
  \item TP tests unitaires (1 groupe)
  \item Premier projet agile sur un groupe de S3
\end{itemize}

\subsubsection{Seconde année}
\begin{itemize}
  \item Initiation à l'outil de gestion de version de code source : <<git>>. Cours TP, TD pour toute la promotion de S2 (4 groupes + 1 groupe décalé)
  \item Initiation à l'artisanat logiciel (code propre, test unitaire, et TDD) : cours, TP, TD pour toute la promotion de S2 (4 groupes + 1 groupe décalé).
  \item Initiation à l'agilité : cours, TD.
  \item Administration système et réseau: cours, TP, TD en semestre décalé (1 groupe)
  \item projet agile avec tous les S3 (4 groupes + 1 groupe décalé)
\end{itemize}

\subsubsection{Troisième année}
\begin{itemize}
  \item Initiation à l'outil de gestion de version de code source git, généralisation aux formations continues
  \item Initiation à l'artisanat logiciel (code propre, test unitaire, et TDD) : généralisation aux formations continues
  \item Initiation à l'agilité : cours, TD.
  \item Généralisation du projet agile à toutes les promotions (Formation initial, formation continue, décalés) sous 2 formes : en début de S3 puis en fin de S4.
\end{itemize}

\subsubsection{Quatrième année}
\begin{itemize}
  \item Initiation à l'outil de gestion de version de code source git FI et FC.
  \item Initiation à l'artisanat logiciel (code propre, test unitaire, et TDD) FI et FC.
  \item Initiation à l'agilité : cours, TD.
  \item Projet agile de début de S3
  \item Projet agile de fin de S4 avec les GEA et de véritables créateurs d'entreprise portés par la BGE.
\end{itemize}

\subsubsection{Cinquième année}
\begin{itemize}
  \item Initiation à l'outil de gestion de version de code source git FI et FC.
  \item Initiation à l'artisanat logiciel (code propre, test unitaire, et TDD) FI et FC.
  \item Initiation à l'agilité : cours, TD.
  \item Projet agile de début de S3
  \item Projet agile de fin de S4
\end{itemize}

\subsubsection{Sixième année}
\begin{itemize}
  \item Initiation à l'artisanat logiciel (code propre, test unitaire, et TDD) FI et FC.
  \item Initiation à l'agilité : cours, TD.
  \item Projet agile de début de S3
  \item Projet agile de fin de S4 en continuité pédagogique des enseignement d'androïd et de javascript.
\end{itemize}

\subsubsection{Projet agile en détail}

Le projet agile a été construit de façon incrémentale, d'abord 1 groupe en S3, puis toute la promotion de S3, l'ensemble des promotions en S3 et S4, intégration des GEA, puis partenariat avec la BGE pour avoir de véritables créateurs d'entreprise.

Ces 2 projets par an que font maintenant les étudiants, couvrent les objectifs pédagogiques suivants : 
\begin{itemize}
\item Replonger rapidement les étudiants dans la programmation après 2 mois de vacances.
\item l'initiation par la pratique à la gestion de projet agile,
\item la cohésion de la promotion dans son ensemble,
\item la mise en application des enseignements du semestre dans un format startup
\item confronter les étudiants à de vrais clients porteurs de projets.
\item mélanger les étudiants d'informatique et de GEA dans un projet commun afin de découvrir concrètement le métier des autres.
\item savoir vendre et défendre un projet devant du public 
\end{itemize}

La soutenance se faisant sous forme de salon avec communiqué de presse, cela apporte une certaine diversité dans les visiteurs, tout en participant au rayonnement de l'IUT auprès du public.

Pour mener à bien cette activité j'ai largement utilisé mon réseau professionnel afin d'impliquer de vrais porteurs de projets. 
J'ai d'ailleurs initié un partenariat entre AUDACE, la BGE Hauts-de-France et l'IUT.

Enfin, le projet agile de fin de S4 est un véritable outil de communication, faire venir des porteurs de projets dans l'IUT durant une semaine et faire venir les journalistes pour le salon de soutenance, c'est un bon moyen de recruter de futurs employeurs et de faire connaître les compétences de nos étudiants.

\section{Activité de recherche / valorisation}

\begin{itemize}
  \item Implication dans la réflexion d'un DU <<Lean startup>>
  \item Implication dans la création d'une licence pro <<Développement mobile>> 
  \item Organisation des Coding goûter : initiation de l'informatique aux enfants entre 5 et 18 ans.
  \item Organisation des Meetup lean startup durant lesquels je retrouve d'anciens étudiants maintenant porteurs de projet et des enseignants chercheurs en phase de création d'entreprise.
  \item Organisation des Meetup Docker
  \item Participation à quelques Startup weekend
  \item Conférencier dans de nombreux événements comme l'agile tour Lille, agile Laval, mixit, lean kanban france, agile open France, agile France, etc.
\end{itemize}

Mes clients étant nombreux et majoritairement sur la région, il n'est pas rare de croiser des étudiants en stage durant l'une ou l'autre de mes visites, c'est l'occasion d'échanger sur le stage, le stagiaire et l'IUT.

\section{Implication dans la vie de l'IUT}

En plus des éléments déjà abordés précédemment comme mon travail sur la collaboration des entreprises avec l'IUT avec les projets agiles, ma participation au rayonnement de l'IUT à travers mes multiples activités je participe activement à la vie de l'IUT.

\begin{itemize}
  \item Je participe avec une double casquette aux journées de l'entrepreneuriat organisée par l'IUT.
  \item Je suis responsable des modules pédagogiques M3301, M3302 et M4106.
  \item J'organise régulièrement des ateliers avec mes collègues autour des sujets de l'artisanat logiciel.
  \item Je participe à la cohérence des enseignements de l'équipe pédagogique en proposant une activité de fin de formation fédératrice d'enseignements et en apportant une coloration artisan du logiciel. Cohérence que l'on retrouve par exemple dans l'usage et l'enseignement des tests unitaires du S1 au S4.
  \item Dans une optique de cohérence des enseignements de l'équipe pédagogique, les enseignements d'Androïd, REST et Javascript sont, depuis cette année, co-construit pour arriver à un socle commun directement utilisable en projet agile.
  \item Mon réseau professionnel fait partie de mes outils de travail et mettre en relation des gens de mon réseau avec des gens de l'IUT pour produire de la valeur fait sens à mes yeux.
\end{itemize}


\section{Bilan}

Que ce soit par les enseignements que j'assure, via la formation des enseignants aux différentes techniques d'artisanat logiciel, par ma charge pédagogique, par mon implication dans un programme pédagogique impliquant de nombreuses matières ou le partage de réseau professionnel, mon investissement dans l'IUT est réel et complet. Et c'est avec plaisir que je renouvellerai mon implication durant les années à venir.

\end{document}

