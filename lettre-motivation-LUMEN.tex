\documentclass{lettre}
\usepackage[utf8]{inputenc}
\usepackage[T1]{fontenc}
\usepackage[frenchb]{babel}
\usepackage{hyperref}
\usepackage{url}
%\setlength{\listindentFB}{5em}
%\usepackage{tango}

\hypersetup{ 
    backref=true, %permet d’ajouter des liens dans…
    pagebackref=true, %…les bibliographies
    hyperindex=true, %ajoute des liens dans les index.
    colorlinks=true, %colorise les liens
    breaklinks=true, %permet le retour à la ligne dans les liens trop longs
    urlcolor=black, %couleur des hyperliens
    linkcolor=black, %couleur des liens internes
    bookmarks=true, %créé des signets pour Acrobat
    bookmarksopen=true,
    pdftitle={}, pdfauthor={}, pdfsubject={}
    %si les signets Acrobat sont créés, %les afficher complètement. %informations apparaissant dans %dans les informations du document %sous Acrobat.
}

\begin{document}

\begin{letter}{ Benoît Demil }

    \address{\textsc{Thomas Clavier}\\45 rue d'Inkermann\\59100 Roubaix}
    \telephone{+33(0) 6 20 81 81 30
    }
    \email{thomas.clavier@univ-lille.fr}
    \nofax
    \lieu{Roubaix}

    \makeatletter
    \def\rule@length{0}
    \makeatother

    \conc{Demande de rattachement au LUMEN}

    \signature{Thomas Clavier}

    \opening{Monsieur,}

    Je suis Thomas Clavier, PAST à l'université de Lille au département informatique de l'IUT. De formation ingénieur, je n'ai malheureusement jamais été rattaché à un laboratoire de recherche et n'ai participé à aucune publication. Cependant je demande le rattachement au LUMEN afin d'une part d'encrer le partenariat entreprises et recherche fondamentale et d'autre part, de poursuivre mon implication dans l'université.

    Dans cette lettre, je vais tenter de répondre à certaines questions que vous pouvez vous poser.

    \textbf{Mon parcours d'entrepreneur}

    \begin{itemize}
        \item 2007, directeur technique d'une PME de 30 personnes exclusivement en télétravail.
        \item 2011, création d'un cabinet de conseil en innovation managériale (https://azae.net)
        \item 2022, création d'un cabinet de conseil et d'accompagnement en organisation d'entreprise (https://aqoba.fr).
        \item Accompagnement de nombreuses startups dans leur phase d'incubation à la Plaine Images.
    \end{itemize}

    À travers ces 2 structures j'ai accompagné la transformation managériale et organisationnelle de quelques entreprises, la majorité du temps en relation directe avec le comité de direction ou le comité executifs. Quelques exemples : Auchan Direct, Société Générale, Enedis, RATP, MSA, EDF, SACEM, Renault, Scaleway, Cityscoot, Mongopay, Atos, Sencrop.

    \textbf{Mon parcours d'enseignant}

    PAST à l'IUT de lille au département informatique depuis 9 ans, j'assure un ensemble de cours mais surtout j'ai monté une grande activité regroupant tous les étudiants d'informatique et une partie des étudiants de GEA : la startupweek (\url{https://startupweek.fr/}). Cette activité d'une semaine a pour objectif :
    \begin{itemize}
        \item de faire connaître le tissu économique des startupers à nos étudiants, 
        \item de leur montrer comment mettre en oeuvre 2 ans d'enseignement académique dans un cas concret et pratique 
        \item de les initier à l'agilité
    \end{itemize}

    Cette activité a été le point d'entrée de ma collaboration avec Valérie François. Au delà de faire collaborer nos départements et de montrer à nos étudiants la complémentarité des formations et des compétences, nous avons commencé à travailler sur les sujets suivants :
    \begin{itemize}
        \item La réalisation d'un ouvrage sur le financement des start-ups. Projet STEFI déposé à l'ANR.
        \item L'étude de cas startupweek que nous souhaitons publier à la centrale pédagogique des IUT, il est d'ailleurs possible de voir une première version de la vidéo que la DOSIMA est venue tourner cette année : https://s.42l.fr/startupweek
    \end{itemize}

    Je passe mon implication dans la création du BUT 2 et 3 qui risque de ne pas intéresser beaucoup le laboratoire.

    \textbf{Les compétences que je peux mettre au service du LUMEN}

    De part ma formation et mon expérience professionnelle, il y a 2 choses que j'affectionne particulièrement, le beau code et l'innovation managériale.
    Sans prendre beaucoup de risque, je peux parier que je pourrais être le plus compétent du labo en site web, R et Python. Je ne suis pas sûr que ce soit l'expertise dont le labo ait le plus besoin, mais ça peut servir.

    Expert reconnu dans l'agilité, je me ferais un plaisir de partager avec vous tout ou partie de cette culture. Que ce soit sur :
    \begin{itemize}
        \item les méthodes (Scrum, SAFe, Kanban, etc. ), 
        \item les pratiques (rituels et ateliers courant comme : daily, rétrospectives, health check, backlog grooming, PI planning, atelier craft, toyota kata, etc.) 
        \item l'état d'esprit ou la philosophie qui la sous tend.
    \end{itemize}

    Pratiquant de longue date l'entreprise libérée et diverses formes de management horizontal je pourrais aussi partager quelques unes de mes compétences sur ces sujets. 

    Enfin je pense que nous partageons une passion : l'innovation managériale. Celle-ci fait intégralement partie de mon ADN et de l'ADN des entreprises que j'ai créées.

    \textbf{Une démarche scientifique non aboutie}

    Je me trompe peut-être, mais pour moi, une démarche scientifique consiste à partir d'un sujet, l'étudier avec rigueur, en tirer des apprentissages, de temps en temps on arrive à en faire un modèle, puis de publier quelques articles sur le sujet. N'ayant jamais eu la nécessité de publier, je me suis régulièrement contenté d'étudier des sujets, en tirer des apprentissages et les appliquer ou les partager sous forme de conférences.

    Voici donc une liste non exhaustives des sujets que j'ai étudiés en profondeur soit pour les appliquer directement dans mon contexte professionel soit pour en faire quelques conférences : 

    \begin{itemize}
        \item Le management et l'organisation du travail dans une entreprise uniquement en télétravail. Mise en application dans mes équipes et création d'une conférence REX.
        \item Le management chez les militaires, en particulier à la Légion étrangère et dans l'OTAN. Interview de plusieurs dizaines de militaires et en particulier du Général Yakovleff (ancien Général en chef de l'OTAN). Production d'une conférence et utilisation régulière en clientèle.
        \item Les entreprises libérées, rencontre de tous les dirigeants membre de l'association MOM21 et d'une dizaine de dirigeants de la région Lilloise pour construire une conférence intitulée "modèle de transformation vers l'entreprise libérée" en ouverture d'une journée dédiée au sujet de l'entreprise libérée organisée par la CCI Nord de France.
        \item Le leadership et une mise en application pratique avec le "Host leadership", étude de publication sur le sujet, rencontre de la Chaire Leadership de l'EDHEC et construction d'une conférence.
        \item Étude de la perméabilité entre la structure organisationnelle de l'entreprise et la structure du code. Conséquence sur la maintenabilité et les notions de dette techniques. De l'importance de l'organisation et du management pour anticiper le scale-up et/ou les pivots des organisations. Entreprises étudiées : blablacar, doctissimo, scaleway, mongopay, Sencrop, Netflix, LinkedIn, Viadeo, deezer, et lego. Mise en application direct chez les clients.
        \item "Indicateur, objectif et management. La sociologie comme clé de lecture", construction d'une conférence pour alerter sur la loi de Campbell/Goodhart.
    \end{itemize}

    \textbf{Les sujets que j'aimerais aborder}

    Il existe deux sujets que j'aimerais travailler dans un avenir proche et qui je pense auraient toute leur place au seins du LUMEN.

    \begin{itemize}
        \item Le financement des start-ups.
        \item Le management dans les start-ups et le changement de management au moment où elles deviennent des scale-ups.
    \end{itemize}

    \textbf{En très résumé}

    Ce que je peux apporter au laboratoire :
    \begin{itemize}
        \item mes compétences techniques
        \item mes connaissances et mon expérience sur le management.
    \end{itemize}

    Ce que le laboratoire pourrait m'apporter :
    \begin{itemize}
        \item de la rigueur et de l'aide pour aller jusqu'au bout d'une démarche scientifique.
        \item une co-construction des modes de management du futur.
    \end{itemize}

    Je suis convaincu que l'industrie doit se nourrir de la recherche fondamentale et que la recherche ne doit pas rester purement académique.

    \closing{Dans l'attente de votre réponse, je vous prie d’agréer, Monsieur, mes sincères salutations.}

\end{letter}
\end{document}

